
\chapter{Finite-tree algorithm}
\label{ch:finite-tree}
In this chapter i introduce the reader to the concepts of the finite-tree algorithm. 

\section{Notation and definitions}

Throughout this thesis, we consider a finite set of matrices \( \mathcal{A} = \{A_1, \dots, A_J\} \) with \( A_j \in \mathbb{R}^{s \times s} \). The computation of the JSR using tree-based methods, requires a structured representation of matrix products.

\subsection*{Encoding of Matrix Products}
The set \( J^n := \{1, \dots, J\}^n \) denotes the collection of all index sequences of length \( n \). For an index sequence \( j = [j_1, \dots, j_n] \in J^n \), the corresponding matrix product is given by
\[
    A_j = A_{j_n} \cdots A_{j_1}.
\]
For \( n = 0 \), we define the identity matrix \( A_j = I \).

\subsection*{Definition of an finite-tree}
\label{def:tree}
An \( (A,G) \)-tree is a structured representation of matrix products used to decompose arbitrary matrix products from $\mathcal{A}$. Lets define it given a set of generator indices \( G = \{g_1, \dots, g_I\} \subseteq J^n \):
\begin{definition}
    A Tree with the following structure: 
\begin{itemize}
    \item The root node contains the identity matrix: \( t_0 = \{I\} \).
    \item The root node is parent of exactly \( J \) children: \( \{A_j P : P \in t\}, j = 1, \dots, J \)
    \item Each node \( t \in T \) is either:
    \begin{itemize}
        \item A leaf (i.e., it has no children),
        \item A parent of exactly \( J \) children: \( \{A_j P : P \in t\}, j = 1, \dots, J \) (positive children),
        \item A parent of arbitrarily many generators: \( \{A_g^n P : P \in t, n \in \mathbb{N}_0\} \) for some \( g \in G \) (negative children).
    \end{itemize}
\end{itemize}
Is called $(\mathcal{A},\mathbf{G})\text{-tree}$
\end{definition}

\subsection*{Covered Nodes}
A node in the tree is called \textit{covered} if it is a subset of one of its ancestors in the tree. Otherwise, it is called \textit{uncovered}. The set of uncovered leaves is denoted as
\[
    \mathcal{L}(T) := \{ L \in t : t \text{ is an uncovered leaf of } T \}.
\]
and called leafage of the tree $T$.

\subsection*{1-Boundedness}
An \( (A,G) \)-tree $T$ is called \textit{1-bounded} with respect to a sub-multiplicative matrix norm \( \| \cdot \| \) if
\[
    \sup_{L \in \mathcal{L}(T)} \| L \| \leq 1.
\]
If a stricter bound holds, i.e., \( \sup_{L \in \mathcal{L}(T)} \| L \| < 1 \), then the tree is called \emph{strictly 1-bounded}.

\subsection*{Possible trees remark}
Since we are only interested in the smallest possible tree (smallest runtime) there are some trees that can be sorted out.
For example if we take a generator directly at the beginning of the tree, per definition for the tree to be 1-bounded also choosing $n = 0$, meaning the generator acts as identity, should yield norms less then 1. But that means the generator could have been left out and all norms are still less than 1, making it unnecessary. Also putting a generator as a leaf is unefficient: either the leaf is 1-bounded and the same argument with $n=0$ from before holds or it is covered by another node. I propose that a generator leaf can only be covered by another generator leaf. That would mean we at least have one 1-bounded generator leaf which is declared inefficient from before. 

\begin{proof}
    Let the encoding of the leaf $t_1$ node be $j = [P,g]$. $P$ stands for prefix and $g$ is the index for the used generator. For it to be covered, there must exist a node $t_2$ with the encoding $[P', g, S]$ where the prefix is different and it also is allowed to have a suffix. Since all products from $t_1$ are in $t_2$ the suffix can only be of the form $S = g^k$. Now choosing $n = 0$ for the generator of the node $t_1$, we get $P = g^k \cdot g^m \cdot P'$ for some $m \in \mathbb{N}$ but that would mean the node at $P'$ has positive and negative children, which is not allowed, forming a contradiction. 
\end{proof}

\section{Structure of the finite-tree algorithm}
We start with the root node, the identity $t_0 = {I}$. Now we build up a tree thats of the form of \ref{alg:tree}.
Possible methods would be breadth-first- or depth-first-search for building up the tree. 
The chosen norm in this case is arbitrary but changes the runtime dramatically. 
If the 1-bounded tree was found, we have proven that there exists a decomposition for every product of matrices from $\mathcal{A}$.

\vspace{1cm}

\begin{algorithm}
    \caption{Finite-tree algorithm}
    \label{alg:tree}
\end{algorithm}

\vspace{1cm}

\begin{theorem}
    If the finite-tree algorithm \ref{alg:tree} terminates the $JSR(\mathcal{A})$ was found. 
\end{theorem}

\begin{proof}
    The leafage of the generated tree is 1-bounded. Now we take an arbitrary product $P \in \mathcal{A}^n$.
    If the profduct is outside the current tree i.e. after applying lexicographic ordering on the encodings of the nodes the encoding of the product is bigger then every other encoding in the tree. 
    That means the structure of $(A,G)$-trees allows us to find one leaf-node that is a valid prefix of the product. Now we have $P = P'L$ with $\lVert L \rVert \le 1$. Since every leaf-node is guaranteed to have at least length 1 (first branches must be $\mathcal{A}$) the length of $P'$ is strictly smaller then the length of $P$. We can repeat that process until $P'^{(k)}$ is within the tree.
    If the product is within the tree, we can generate a polynomial $p(k)$ that is monotone and an upper bound on the norms of products within the tree, where $k$ is the length of the product. 
    For that we have to analyse the Jordan-Normalform of the products. Since the spectral radius of every matrix in $\mathcal{A}$ and of every generator is less than 1, we now the growth of potencies of the Jordan-blocks is bounded by a polynomial in the order of the size of the Jordan-block. Just taking a maximum over all factors of all such polynomials we generate a polynomial that bounds every product within the tree according to the length of the product.  
    So every product $P \in \mathcal{A}^n$ is of the form $P = S \cdot P_k \cdot ... \cdot P_1$ where $\lVert P_i \rVert \le 1$ and $\lVert S \rVert \le p(k)$. We imply: 
    $$JSR(\mathcal{A}) = \lim_k \max_{P \in \mathcal{A}^k} \lVert P \rVert ^{\frac{1}{k}} \leq \lim_k p(k) ^{\frac{1}{k}}  = 1$$
    The detailed proof can be read in \citep{mollerTreebasedApproachJoint2014}. 
\end{proof}

\section{Efficiency results}

\begin{theorem}
    \label{thm:same_solution}
    If the invariant-polytope algorithm terminates so does the finite-tree algorithm.
\end{theorem}

\begin{proof}
    The proof can be seen in \citep{mollerTreebasedApproachJoint2014}.
\end{proof}

\begin{theorem}
    The solution space of the finite-tree algorithm is strictly bigger then the one from the invariant-polytope algorithm. 
\end{theorem}

\begin{proof}
    From theorem \ref{thm:same_solution} we know that the finite-tree algorithm always terminates if the invariant-polytope algorithm does. 
    Lets view an example for when the invariant-polytope algorithm cannot terminate due to a missing spectral-gap. But the finite-tree algorithm can find $JSR(\mathcal{A})$ nonetheless. 
\end{proof}

\subsection*{Tree generation}
The runtime does not only depend on the chosen norm but also on the use of the generators. 
The idea is that the generators come closest to the maximal growth rate so using them will maximize the norm and the subsequent branches can almost only decrease the norms eventually so that the 1-boundedness can be checked earlier. 

