
\chapter{Invariant-polytope algorithm}
\label{ch:inv.poly}	

In this chapter we bring our interest to the underlying invariant-polytope algorithm. 
One result about JSR computation, that every irreducable family possesses an invariant norm is helpful. We observe that there always exists an norm that is in some sense extremal. 

\section{Extremal norms}
\label{sec:extremal-norms}

A norm is called invariant if it is preserved under the action of the family of matrices. That is, for all $A_j \in \mathcal{A}$ and $x \in \mathbb{R}^d$ we have $\lVert A_j x \rVert = \lVert x \rVert$.

\begin{theorem}
    Every irreducible family $\mathcal{A}$ possesses an invariant norm.
\end{theorem}

\begin{definition}
    A norm $\lVert \cdot \rVert$ is called extremal for a family of matrices $\mathcal{A}$ if $\lVert A_j x \rVert \leq \rho(\mathcal{A}) \lVert x \rVert$ for all $x \in \mathbb{R}^d$ and $A_j \in \mathcal{A}$.
\end{definition}

Every invariant norm is extremal

\section{Invariant polytope norms}
For every polytope there exists a \emph{Minkowski norm}

\section{Structure of the invariant-polytope algorithm}
After the preprocessing is over we start with the leading eigenvector of the calculated candidate. 
Now every product with matrices from $\mathcal{A}$ and vertices from $V$ is added if the current polytope-norm is bigger or equal to 1.

\vspace{1cm}

\FloatBarrier

\begin{algorithm}
\caption{invariant-polytope algorithm}
\label{alg:exact}
\begin{algorithmic}

\State V := $\{v_1, \cdots, v_M\}$
\State $V_{\text{new}} \gets V$
\While {$V_{\text{new}} \ne \emptyset$}
\State $V_{\text{rem}} \gets V_{\text{new}}$
\State $V_{\text{new}} \gets \emptyset$
\For {$v \in V_{\text{rem}}$}

\For {$A \in \mathcal{A}$}
\If {$\lVert Av \rVert_{\text{co}_{\text{s}}(V)} \geq 1$}
\State $V \gets V \cup Av$
\State $V_{\text{new}} \gets V_{\text{new}} \cup Av$
\EndIf
\EndFor
\EndFor
\EndWhile \\
\Return $\text{co}_{\text{s}}(V)$ \\
\end{algorithmic} 
\end{algorithm}

\FloatBarrier

\vspace{2cm}

\section{Stopping criterions}
The runtime of the algorithm \ref{alg:exact} is not finite. 
Suitable conditions for stopping or recalculating a candidate have to be put in place for the algorithm to have a finite runtime. 
Of course a bare minimum of an max iteration trigger is implemented. 
The paper \citep{guglielmiExactComputationJoint2011} promises at least good bounds on the real JSR value while also proposing a stopping criterion thats based on eigenplanes. This criterion also generates a new candidate if the last wasnt sufficient after finite time.
The last criterion is ... [dont remember right now].
In the following section the needed theory from \citep{guglielmiExactComputationJoint2011} is restated. 

\section{Termination conditions}
The invariant-polytope algorithm is very efficient but has its caviat. It only is guaranteed to terminate in finite time if the leading eigenvector of the chosen candidate is unique and simple. 
In the following i will present the according proofs. 

\section{Rebalancing and added starting vertices}
Three years after publishing the fundamental algorithm, the creators released a new paper on rebalancing multiple s.m.ps as well as starting with some extra vertices so the polytope is conditioned better (not as flat).

\section{Eigenvector cases}
If the eigenvector is complex a different norm must be used a so called complex symmetric polytope norm. 
Also in the case of nonnegative matrices a so called cone-norm is used instead of the polytope-norm since its way more efficient. 


