%%%%%%%%%%%%%%%%%%%%%%%%%%%%%%%%%%%%%%%%%%%%%%%%%%%%%%%%%%%%%%%%%%%%%%%%%%%%%%
% Die Zusammenfassung/Der Abstract scheint nicht im Inhaltsverzeichnis auf.
%%%%%%%%%%%%%%%%%%%%%%%%%%%%%%%%%%%%%%%%%%%%%%%%%%%%%%%%%%%%%%%%%%%%%%%%%%%%%%
\chapter*{Zusammenfassung}

Diese Arbeit präsentiert einen hybriden Ansatz zur Berechnung des gemeinsamen Spektralradius (JSR) einer endlichen Menge von Matrizen. Die vorgeschlagene Methode integriert die Invariante-Polytop und Endlicher-Baum Algorithmen, um deren jeweilige Stärken zu nutzen. Die allgemeine Anwendbarkeit des baumbasierten Ansatzes wird mit der Effizienz der Polytop-basierten Techniken kombiniert. Der entwickelte Algorithmus konstruiert ein Baumdarstellung und überprüft maximale Produkte mithilfe angepasster Normen, wodurch eine Terminierung in Fällen ermöglicht wird, in denen klassische Methoden an ihre Grenzen stoßen (Bedingungen an Terminierung oder Laufzeit). Theoretische Garantien werden bereitgestellt, um Korrektheit und Effizienz zu belegen. Es werden auch JSR-Berechnungen für Matrizenmengen, bei denen bestehende Methoden scheitern oder ineffizient sind, demonstriert.

\vspace{1.3em} 					% um den Abstand zum Text richtig zu stellen
\section*{\huge Abstract} 	% Trick! Abstract auf gleicher Seite.
\vspace{1.3em} 					% um den Abstand zum Text richtig zu stellen

This thesis presents a hybrid approach for computing the joint spectral radius (JSR) of a finite set of matrices. The proposed method integrates the invariant polytope algorithm and the finite tree algorithm to leverage their respective strengths — integrating the vast applicability of the finite-tree-based approach with the efficiency of polytope-based techniques. The developed algorithm constructs a structured tree representation and verifies spectrum-maximizing products using adapted norms, enabling termination in cases where classical methods take too long or fail. Theoretical guarantees are provided to demonstrate correctness and efficiency. \\ \\
\textbf{Keywords:} Joint spectral radius, hybrid algorithm, invariant polytope, finite tree method, matrix norms.

%%%%%%%%%%%%%%%%%%%%%%%%%%%%%%%%%%%%%%%%%%%%%%%%%%%%%%%%%%%%%%%%%%%%%%%%%%%%%%