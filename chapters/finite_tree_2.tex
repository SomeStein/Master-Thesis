\chapter{A Tree-Based Method for Joint Spectral Radius Computation}

\section{Motivation and Background}

The \emph{joint spectral radius} (JSR) of a finite set of matrices $\mathcal{A} = \{A_1, \dots, A_m\} \subset \mathbb{C}^{d \times d}$ is a generalization of the spectral radius for a single matrix. It quantifies the maximal asymptotic growth rate of all infinite products formed from elements of $\mathcal{A}$:
\[
\hat{\rho}(\mathcal{A}) := \limsup_{k \to \infty} \max_{I \in \{1, \dots, m\}^k} \|A_{i_k} \cdots A_{i_1}\|^{1/k},
\]
independently of the chosen matrix norm.

The JSR arises in diverse applications, such as wavelet theory, control theory, and subdivision schemes. However, its exact computation is notoriously difficult—it is generally undecidable and NP-hard to approximate.

One known sufficient condition is the \emph{finiteness property} (FP), which holds if a finite product achieves the JSR:
\[
\exists J \in \{1, \dots, m\}^k \text{ such that } \hat{\rho}(\mathcal{A}) = \rho(A_J)^{1/k}.
\]
This work builds on and generalizes algorithms that aim to verify this property using a novel tree-based approach.

\section{Tree-Based Algorithmic Framework}

The key idea is to construct a \emph{finite tree} in which nodes represent \emph{sets} of matrix products (rather than individual products). This allows verification of the FP even when reducibility or lack of asymptotic simplicity prohibits other approaches.

\subsection{Encoding Matrix Products}

Let $J_1, \dots, J_n$ be index vectors representing \emph{generators}, i.e., products $A_{J_i}$ with $\rho(A_{J_i}) = 1$. We then define a tree $\mathcal{T}$ whose nodes correspond to sequences $K = [k_1, \dots, k_\ell]$ with $k_i \in \{-n, \dots, -1, 1, \dots, m\}$. Positive entries refer to individual matrices $A_i$, and negative entries to powers of generator products, i.e., $A_{-j} = \{A_{J_j}^k : k \in \mathbb{N}_0\}$.

The matrix product set associated with a node $K$ is defined recursively:
\[
A_K = A_{k_\ell} \cdots A_{k_1},
\]
where multiplication of sets is defined as the set of all resulting matrix products.

\subsection{Node Classification}

Each node $K$ is classified based on:
\begin{itemize}
  \item \textbf{1-boundedness}: $\sup\{\|A\| : A \in A_K\} \leq 1$.
  \item \textbf{Coveredness}: There exists a prefix $P$ of $K$ such that $A_K \subset A_P$, with the complementary suffix being a nonempty positive index sequence.
\end{itemize}

\subsection{Main Theorem and Verification Strategy}

\begin{theorem}[Möller \& Reif]
Let $\mathcal{A}$ be a finite set of matrices, and $J$ a generator set with $\max_{J_i \in J} \rho(A_{J_i}) = 1$. If there exists a finite subtree $\mathcal{T}^* \subset \mathcal{T}$ such that:
\begin{enumerate}
  \item The root $\emptyset$ has exactly $m$ positive children,
  \item Every leaf is either 1-bounded or covered,
  \item Every internal node has either $m$ positive or arbitrary negative children,
\end{enumerate}
then $\hat{\rho}(\mathcal{A}) = 1$.
\end{theorem}

This result allows numerical or symbolic verification of the FP by searching for a tree $\mathcal{T}^*$ satisfying the above.

\subsection{Algorithmic Construction}

The tree is built via a depth-first traversal:
\begin{enumerate}
  \item Start with root node $\emptyset$.
  \item For each node, append $m$ positive children unless the node is covered or 1-bounded.
  \item Optionally, add negative children corresponding to powers of generators if it helps cover deeper branches.
  \item Terminate when all leaves satisfy the theorem's conditions.
\end{enumerate}

The norm used affects the tree structure. For irreducible families with a \emph{spectral gap at 1}, an appropriate norm guarantees termination.

\section{Remarks and Extensions}

The method is applicable even in cases where:
\begin{itemize}
  \item $\mathcal{A}$ is reducible,
  \item multiple generator sets are needed,
  \item no dominant eigenvalue exists (i.e., no spectral gap).
\end{itemize}

This makes it a versatile building block for hybrid strategies combining polytope norm methods and symbolic exploration of product structures. Our implementation extends this algorithm by allowing generators with spectral radius less than 1 to accelerate termination.