
\chapter{Diskussion}
\label{ch:diskussion}

Die Diskussion schließt die Arbeit ab. Im Folgenden gebe ich noch ein Beispiel für eine Tabelle, die unter Zuhilfenahme des Paketes \texttt{booktabs} hübschere Ergebnisse liefert, als die \LaTeX-üblichen Tabellen.

% Die folgende Tabelle basiert auf dem Paket {booktabs}. Sie sieht den
% Tabellen in echten Publikationen sehr ähnlich. Jedoch sind hier senkrechte
% Linien nicht ordentlich realisierbar (wie sie in Publikationen auch nie
% vorkommen). Um Tabellen mit senkrechten Linien zu erzeugen, empfiehlt sich
% weiterhin die im Kurs besprochene Lösung mit der Angabe \hline für horizontale
% Linien, statt den hier gezeigten \toprule, \midrule und \bottomrule Befehlen.

\begin{table}[ht]
\centering
\caption[Kurzname Tabelle]{Eine Tabelle zur Verdeutlichung von Tabellen in \LaTeX. Es handelt sich um eine hübschere Tabelle, als im Kurs besprochen, die dem Aussehen von Tabellen in echten Publikationen sehr nahe kommt.}
\begin{tabular}{clp{6cm}}
\addlinespace					% fügt einen kleinen Abstand hinzu
\toprule						% Top-Linie bei einer Tabelle
Buchstabe   &	Wort	&	Satz			\\
\midrule						% dünnere Linie für Tabellen
A			&	Affe	&	Der Affe sitzt auf dem Baum und traut sich nicht herunter	\\
B			&	Biene	&	Die Biene fliegt Kreise um ihren Stock, und ihr wird dennoch nicht schwindelig \\
K			&	Karibu	&	Das Karibu ist nicht wie du! \\
\bottomrule						% Schlußlinie bei einer Tabelle
\end{tabular}
\label{tab:Buchstaben}		    % für einen Textverweis mittels \ref{tab:Buchstaben}
\end{table}


%%%%%%%%%%%%%%%%%%%%%%%%%%%%%%%%%%%%%%%%%%%%%%%%%%%%%%%%%%%%%%%%%%%%%%%%%%%%%%
\chapter{Schlussfolgerung}
\label{ch:schlussfolgerung}

Die Conclusion sollte alles oben erarbeitete Zusammenfassen und zu einem Ergebnis bringen.

Zum Testen kommt hier noch ein Literaturzitat. Wie in \cite{Dockner2010} bereits anschaulich erklärt, bietet \LaTeX\ gerade für Master- oder PhD-Arbeiten enormes Potential.